\documentclass{article}



\usepackage[utf8]{inputenc}

%\usepackage{natbib}
\usepackage{hyperref}

\usepackage{graphicx}

\usepackage[colorinlistoftodos]{todonotes}

\usepackage{parskip}
\setlength{\parskip}{10pt} 

\usepackage{tikz}
\usetikzlibrary{arrows, decorations.markings}

\usepackage{chngcntr}
\counterwithout{figure}{section}



\begin{document}


\begin{titlepage}
  

\centering
  
  
{\scshape\LARGE Master thesis project proposal\\}
  
\vspace{0.5cm}
  
{\huge\bfseries Injection of Information to Embodied Agents\\}

  
\vspace{2cm}
  
{\Large Yasmeen Emampoor (gusemampya@student.gu.se)\\}
  
\vspace{1.0cm}
  
{\large Suggested Supervisor: Simon Dobnik (FLoV)\\}
  
\vspace{1.5cm}
  
  
{\large Relevant completed courses:\par}
{\itshape (DIT381, Algorithms for Machine Learning and Inference)\\ (DIT866, Applied Machine Learning) \\ (DIT869, Deep Machine Learning)}
  
\vspace{1.5cm}
  
\vfill

\vfill
  
{\large \today\\} 


\end{titlepage}
\section{Introduction}
Robots have the potential to be useful in many fields–they can perform tasks
that are dangerous for humans, as well as tasks that are simply tedious. These
robots are often operated based on pre-defined actions, or remotely. The potential increases in efficiency from being able to give a goal/task in natural language and have the robot interpret and determine how to carry out this task are huge. % SD 2021-02-16 09:03:48 +0100: Also add answer question answering to cover both scenarios.
This would make robots more accessible generally, since it would reduce the
learning curve for operating them. A patient in a hospital would likely find it much easier to simply tell the robot 'I need water', rather than having to find the 'water' option on some sort of touchscreen interface. It also would support situations where the instructions being given are too complex for this kind of interface--one example of this is human-robot teaming, in which groups of humans and robots cooperate to achieve tasks in situations such as disaster scenarios\cite{Kruijff-Korbayova:2015aa}. Communication in natural language can facilitate organisation where all of the humans are working with all of the robots, rather than an individual human having control over one or many robots. Situations like this require complex understanding of natural language instructions. 
% SD 2020-12-11 21:10:09 +0100: There may be cases where even more complex understanding of the natural language instructions is required which would be very complex for the person so "describe" using touch screen interface. For example, human-robot teaming: https://link.springer.com/article/10.1007/s13218-015-0352-5 \cite{Kruijff-Korbayova:2015aa} and There is a lot of research in this area how robots and humans could work together, see for example the references to this paper I found https://arxiv.org/pdf/2009.00288.pdf


% SD 2021-02-16 09:04:43 +0100: Present some examples of dialogues that we would like to model here.
For a household agent, it would be great to be able to interact with a robot in a similar way to how one would with another person. 
\begin{verbatim}
Human: Go get my computer from the office. 
Robot navigates to the office, finds the computer, brings it back. 
Human: Were my keys in the office?
Robot: I don't know, I'll go check. 
Robot navigates to the office, finds the keys. Returns.
Robot: Yes, they are on the bookshelf. 
Human: Thanks. 
\end{verbatim}


For this project, I'll be considering a simpler dialogue stub: the human asks a question, and the robot answers. 
\begin{verbatim}
Q: Where is the computer?
A: In the office. 
\end{verbatim}

\section{Context}
\subsection{Navigation}
Navigation is a common task for robots, a subset of instruction following; applications for robots include moving through a space, recording or streaming video back to a human, and collecting objects in a space to deliver to someone in a different space. The basic task can be split into two main challenges: self-localization and mapping\cite{Wallgrun:2007zr}. However, this leaves out a key point. How does the robot know where to go? For ease of interaction with humans, natural language instructions are often ideal, since the human interacting with the robot can give directions or instructions in the same way that they are already comfortable with. This means that the robot needs to be able to connect that natural language to an object or action, a task known as grounding. 
% SD 2021-02-16 09:06:20 +0100: Distinguish between navigation where the robot only relies on the observations to find a path trough its environment (described here) and navigation where natural language component is added and the robot needs to interpret natural language instructions as well as navigate in its environment.

% SD 2020-12-11 21:37:27 +0100: Here you could expand that there is active research on how humans navigate and reason in space. Perhaps a paper by Diedrich Wolter: https://www.sfbtr8.spatial-cognition.de/project/r3/material/QSR_Challenges.pdf \cite{Wallgrun:2007zr}

\subsection{Grounding}
%A key challenge in creating robots that can interact with humans in natural language is grounding--connecting an external object to a word or phrase. 
The challenge of grounding is key in creating robots that can interact with humans in natural language. The robot must be able to link objects and actions to words. For example, what does 'turn' mean? Grounding is a challenge in a variety of tasks, such as image captioning, where models need to produce labels for both the objects in the image and the objects' interactions with each other in natural language\cite{karpathy2014captioning}.

% SD 2020-12-11 21:25:12 +0100: Actually, the notion of grounding language in perception has been explored in robotics first and then moved into image captioning which is a much younger task: https://canvas.gu.se/courses/36167/assignments/75554 and https://canvas.gu.se/courses/36167/assignments/76897

Image captioning is a fairly new task, but grounding in the field of robotics has been an area of active research for a long time. In 2002, Lauria et al. presented a project in which a robot was given natural language instructions, which were mapped to procedures of identified primitives, such as 'turn'\cite{lauria2002}. More recently, Hermann et al. presented a simulated embodied agent learning grounded language through a combination of unsupervised and reinforcement learning, with minimal prior knowledge\cite{hermann2017grounded}. Another approach with minimal prior knowledge was presented by Thomason et al., where the robot learns through dialogue with a human\cite{thomason2019grounded}.


%\subsubsection{Dialogue}
%Navigation based on dialogue is a current area of active research--datasets for a dialogue based navigation tasks are often collected as human-human dialogues, which are then used to train agents in replay or supervised learning scenarios. 
%
%Cooperative Vision-and-Dialog Navigation is a dataset of over 2000 navigation dialogues\cite{thomason2019visionanddialog}. It was collected via Mechanical Turk crowd-sourcing; pairs used the Matterport3D simulator and a chat interface, with one person, the ’oracle’, able to see the ideal moves for the navigation task, giving natural language instructions to the other person, the ’navigator'. The navigator could ask clarifying questions. The oracle was also shown the navigator's current visual frame. 
%
%A similar but smaller dataset is RobotSlang, consisting of 169 dialogues between a commander referencing a static map and a human driver only able to see the camera view of the robot they were controlling\cite{robotslang}. The commander relayed instructions to the driver, based on their understanding of where the robot was. The driver was able to ask clarifying questions, such as where exactly to turn, and the commander could periodically ask localization questions, such as what color wall the driver could see.
%
%There has also been a lot of research on how humans actually communicate about directions. Humans have a number of strategies for communicating about routes, and which strategy someone will choose to use is often not immediately obvious\cite{Tenbrink:2010qf}. They may also choose to use multiple methods in conjunction. These strategies can relate to what perspective directions are given from or how much detail to give initially, among other things. Another point of consideration is how humans automatically adjust their strategies when interacting with a computer. Tenbrink et al. found that people gave generally sparser commands when interacting with a computer system than they did interacting with another human, even when not given instructions to do so\cite{Tenbrink:2010qf}. 
% SD 2020-12-11 21:42:35 +0100: Similarly, a lot of research has been done on the language side and dialogue. How humans generate descriptions in conversation that are understandable by conversational partners? What landmarks will they use? How will they structure the route description: it is unlikely that they will give a complex description (as you would get from Google straight away). The description will be dependent on who the conversational partner is and what task they are involved in which further complicated computational modelling on the natural language processing side. Perhaps, \cite{Tenbrink:2010qf,Fellner:2017aa,Janarthanam:2012aa}

\subsection{Visual Question Answering}
Visual Question Answering is a task in which an agent must answer a question based on an image; these questions are "free-form and open-ended"\cite{vqa_2015}. This task differs from image captioning in that answering these questions requires specific information, rather than general identification of things/activities in the image. There is a huge variety in possible question types in this task, including object detection ("how many...?"), activity recognition ("Is this person doing ...?"), and knowledge-base reasoning ("what is ... made of?"). 
VQA is a somewhat unusual natural language processing task in that it is often implemented as a classification task, with the agent choosing from a large number of potential answers, rather than generating an answer. This means that the success of VQA can often be measured as accuracy to the ground truth answer. However, VQA can also be implemented as a generation task, for which other measures, like BLEU, which compares a generated output to one or more good human outputs, can be used\cite{bleu}.

\subsection{Embodied Question Answering}
Embodied Question Answering combines the navigation and VQA tasks into one: the embodied agent must navigate to find the object that the question refers to\cite{embodiedqa}. For example, for the question 'What color is the refrigerator?' the agent must identify where the fridge is most likely to be, the kitchen, and navigate there before identifying the fridge and answering the color. This task has been expanded to Multi-Target Embodied Question Answering, in which questions can include multiple targets, allowing for comparison questions such as 'is the oven in the kitchen the same color as the sink in the bathroom?'\cite{eqa_multitarget}. Current approaches for this task use templated approaches for generating questions. The EQA dataset includes nine question types: location, color, color\_room, preposition, existence, logical, count, room\_count, and distance, though the EQA-V1 dataset, used for the experiments by Das et.al includes only the first five question types\cite{embodiedqa}. The MT-EQA dataset adds six comparison question types\cite{eqa_multitarget}.

One concern with EQA models is how much they actually incorporate the visual input in determining an answer. Anand et al conducted an experiment on the EQAv1 dataset that found equivalent to slightly better performance on the question answering task using simple question only models with no visual input\cite{blindfolded}. This is interesting in that it suggests that the models are doing a good job learning common sense knowledge from the textual content, but is a problem in this particular task, since it means that the agent is not actually adapting its answers to the specific situation. 

% SD 2021-02-16 09:16:52 +0100: But the templates are only to create a dataset. The EQA task is then learned with ML and the answers are generated or rather classified.

% SD 2020-12-11 21:42:35 +0100: Similarly, a lot of research has been done on the language side and dialogue. How humans generate descriptions in conversation that are understandable by conversational partners? What landmarks will they use? How will they structure the route description: it is unlikely that they will give a complex description (as you would get from Google straight away). The description will be dependent on who the conversational partner is and what task they are involved in which further complicated computational modelling on the natural language processing side. Perhaps, \cite{Tenbrink:2010qf,Fellner:2017aa,Janarthanam:2012aa}

% SD 2021-02-16 09:17:49 +0100: VQA and E-VQA tasks are actually simplifications of dialogue (which in general would contain several more question types). These tasks have developed with end-to-end deep learning. Modelling dialogies with robots goes way back to the Shaky and Flaky and Winograd's Shrdlu but where language was represented with a rule-based system. Some references here: Winograd:1972, Kruijff:2007, Dobnik:2009dz Hence, it would be nice to conclude here with the previous section on dialogue that is now commented out above.

VQA and EQA can be seen as simplified dialogue tasks. They contain a limited number of question types, and the agent sticks to answering, rather than employing more sophisticated strategies, such as asking questions to acquire more information when it is unsure. One interesting note about dialogue interactions between humans and robots is that humans automatically adjust their strategies when interacting with a computer. Tenbrink et al found that people gave generally sparser commands when interacting with a computer system than they did interacting with another human, even when not given instructions to do so\cite{Tenbrink:2010qf}. 

\subsection{Simulation}
Working with embodied agents is resource intensive and makes reproducibility difficult to impossible, so simulation is beneficial for research. Within simulation, environments can be kept consistent, allowing for both reproducibility of an experiment and for comparison of different systems/methods. Simulation also allows for the reuse of datasets of human descriptions or labels, which are time-consuming and expensive to produce.

AI Habitat is a simulation platform for working with embodied AI\cite{habitat19iccv}. It consists of two parts, Habitat-Sim, the 3D simulator, and Habitat-Lab, the library for embodied AI development. Habitat's current main focus is navigation, mainly through indoor spaces, but there is some ability for object interactions--for example moving a chair from one point to another. Habitat-PyRobot Bridge is a library, written by members of the Habitat team, to support the transfer of a simulated agent in Habitat to a physical robot\cite{Kadian_2020}. Various scene datasets are supported by Habitat, the largest of which is Matterport3D, a dataset of real interiors with human annotation of objects\cite{matterport}.

% SD 2020-12-11 21:35:47 +0100: You could also expand the point about reproducibility earlier that working with simulation environments has benefits for research since different systems can be compared, and datasets of human descriptions (which are expensive to obtain) reused.

\section{Goals and Challenges}

For this project, the overarching question is: how can information learned in other contexts be used to improve performance of an agent in an embodied question answering task? % SD 2020-12-11 22:04:22 +0100: Given this research quetsion it is good that you give more background about the challenges for navigation instructions as I commeneted above.

Since EQA consists of multiple, modular, tasks, it seems like a good candidate for transfer learning. For the task of identifying where to go to find an object, humans use a lot of information that they've learned from other contexts, such as, what room people typically use an item in. Identification of colors can also be learned in non-embodied context, so that would be another place for pre-training to give the agent prior knowledge. 

In Das et. al, they report the agent entering the target room 65\% of the time when the agent is spawned 10 actions away from the target, and 52\% when the agent is spawned 50 actions away\cite{embodiedqa}. This shows a place for improvement. The agent is using visual cues to determine where to go: for example, cars are usually outside, and if it sees a fence through a door, the fence is probably outside. To be able to answer questions, the agent learns labels that it associates with sets of visual cues. These are all learned by training on a large dataset, but this may not be the most efficient or complete method to learn these labels. Initially giving the agent some guidelines, like the ones people use when looking around them, has the potential to make this learning process more complete. 'Stoves are in kitchens', could be given to the agent, and then once the agent has identified a stove, it can know where it is with a high level of confidence. A knowledge graph that connects these labels can give the agent a shortcut to understanding that if it sees a refrigerator through a door, it is likely moving in the correct direction to find a stove. The injection of information from a knowledge graph like that described by Kruijff could improve the performance of the planner\cite{Kruijff:2007}. 

% SD 2021-02-16 09:32:02 +0100: We could investigate the the sucess of VQA when the agent is trained only with information from the domain/environment. Then we progressively add information from different domains in the form of pre-trained feature encoders (e.g. visual features from object classifiers trained on images or word embeddings and language models encoding information about the world) or top-down feature encoders (e.g. to represent geometric information between the embedded encoders). The goal is to show that a system utilising such external knowlede perform better over the systems that have been trained on only the task specific knowledge in an end-to-end fashion.
One challenge is that concrete percentages of improved performance are difficult in NLP. Performance is very dependent on the dataset, and the improvements are often very small. Statistical tests can be run to determine if small improvements are actually statistically significant.
% SD 2021-02-16 09:46:10 +0100: Koen also raises an issue of the concrete % of improved performance. This is hard in NLP. It seems to be easy to achieve large improvements up tom 70% but then improvements become small, some papers are claiming improvements that are less that 1%. This depends on the task and really on a dataset (whose context in which it was collected effectively defines a new variation of a task). The improvements are not linear. We could also run statistical tests to compare the results if these are small.

%Also, datasets for dialogue navigation take a long time to collect, since humans are generating the descriptions in real time, and humans are not necessarily very efficient. For an unsupervised learning approach, several thousand route instructions would be needed. % SD 2020-12-11 21:58:38 +0100: The bottle-neck is mainly due to human descriptions. These take time and several thousand route instructions would be required for an unsupervised approach.
%Because of this, many navigation datasets are small, which limits the agent's knowledge, so being able to use disparate datasets would be beneficial. For example, maybe an agent could learn colors and positions from a dialogue about items on a table, and then would be able to apply that knowledge from the beginning when presented with dialogue navigating through a house. An agent could be trained on a corpora of dialogue from a different domain, and from this learn about human interaction strategies. The system could then be fine tuned on dialogue from the navigation domain, and adapt the knowledge about interaction to understanding route instructions. 

% SD 2020-12-11 22:06:15 +0100: We could train the system on corpora of dialogue from a different domain and then fine tune it on this domain. This way the system would already know something about the interaction strategies of humans; the leanring would then be adapting this knowledge to the route instruction scenario.

%Another use for transfer learning is in the fact that a goal within this research is for these human-human dialogues to become human-robot dialogues, more flexible than those described in \cite{thomason2019grounded}. \todo{add mention of EQA templating approach} Transfer learning could aid in teaching the robot grammar and vocabulary, rather than limiting its knowledge of language to the navigation dialogue, making it easier to create a robot that could generate its own responses to instructions. 

\section{Approach}
I plan to use Habitat to simulate my navigation agents and environment, and use the EQA baseline available in habitat-lab as a starting point. The baseline consists of a feature extraction CNN, a VQA model, and a Navigation model (which consists of a planner and a controller), all of which can be trained independently. (Since it is modular, sections could also be replaced.) Task and scene datasets for Matterport3D are also available. Code for generating EQA-v1 questions is also available, but designed for use with SUNCG, % SD 2021-02-16 09:38:08 +0100: explain what SUNCG is and reference
which is no longer supported, and so would require modification to be useable\cite{embodiedqa}\cite{suncg}. SUNCG is a large dataset of 3D indoor scenes, similar to Matterport3D, with the caveat that SUNCG scenes were designed, whereas the Matterport3D scenes come from real interiors. I will need to identify potential datasets or knowledge bases for background/external knowledge such as navigation datasets in other scenarios (outside, for example), static VQA datasets, or a knowledge graph of information about rooms and contents. 

% SD 2021-02-16 09:38:57 +0100: For datasets see my previous comment.

Next I will train the habitat-baseline EQA, followed by a blindfolded baseline to examine the effectiveness of the VQA model. If the model is relying very heavily on the questions, giving this module more information about language would likely be unhelpful, and improvement of the feature extraction CNN would be a better avenue, such as through transfer learning feature weights learned on larger image datasets. Another potential avenue for transfer learning in regards to the VQA model would be to use weights from CLEVR, which trains on shapes, colors, and positions disconnected from a real world context, since all of the images are of basic 3D shapes\cite{clevr}. This might help the system to focus more on the visual content. Determining whether the system is successfully using the visual content can be determined through comparison to the blindfolded baseline. 

Improvement to the navigator will obviously be more effective in improving the system overall if the system is actually considering the visual content, but if it is, then the navigator is another target for information injection. If an available knowledge graph about house contents can be found, then the next experiment would be to give the planner access to this information. 

Overall performance can be calculated as simple accuracy with regards to the ground-truth answers, since the question answering model is classification across 172 answers. However, since the EQA model actually consists of 3 separate models, they can also be evaluated separately. For navigation, Das et al identify a number of evaluation metrics. For distance they used the final distance to the target, the change in distance to the target over the episode, and the minimum distance to the target at any point in the episode. They also identified the percentage of questions % SD 2021-02-16 09:39:45 +0100: instructions?
% YE 2021-02-17: I'm leaving this as questions since this is the approach section, and not currently planning to expand to other instructions
where the agent ended in the room containing the target, the percentage in which the agent entered the room with the target at any point during the episode, and the percentage of episodes in which the agent stopped to answer the question before the maximum number of actions was reached. I plan to use both ground-truth accuracy and percentages of the agent entering and ending in the correct room as my main measures, but will also be looking at the other distance metrics. 

% SD 2021-02-16 09:41:17 +0100: VQA tasks have in majority cases been taken as a classification of answers which is strange for NLP. The answer should really be generated as a sequence of words. There are a few approaches now that do this (which is of course more complicated). Then evaluation becomes slightly more complicated also, normally measures such as BLEU, METEOR and ROUGE are used to do this.

%datasets of learning information like position in non-navigation scenarios ('what is to the left of the blue ball in this picture?', for example), or general natural language dialogues.

%and the Cooperative Vision-and-Dialogue Navigation set as the main dataset for training and testing of the agents. 

%I will also need to determine what models have been used for navigation agents, and what kinds of representations those models learn and use. These representations will affect what kinds of background knowledge could help the model--a mainly word based representation might benefit from pre-training with general natural language dialogues, a mainly visual one probably wouldn't.

%Next would be to build and train a model for object navigation solely on the Cooperative Vision-and-Dialogue Navigation dataset, to compare with models incorporating background information. After this, I would need to build and train models incorporating background information, and compare them. 

%Generated descriptions can be evaluated based on automatic metrics such as BLEU and CIDEr, as well as via crowdsourcing on platforms such as Mechanical Turk. The performance of the agents at the navigation task can be expressed as the number of steps the agent took over the shortest path (of course multiplied by if the robot ever found the object). 

% SD 2020-12-11 22:08:22 +0100: Looks good! Good luck!


\bibliographystyle{ieeetr}
\bibliography{references}

\end{document}

%%% Local Variables:
%%% mode: latex
%%% TeX-master: t
%%% End:
