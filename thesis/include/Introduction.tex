% CREATED BY DAVID FRISK, 2016
\chapter{Introduction}
\todo{pulled from proposal, rework}
Robots have the potential to be useful in many fields–they can perform tasks that are dangerous for humans, as well as tasks that are simply tedious. These robots are often operated based on pre-defined actions, or remotely. The potential increases in efficiency from being able to give a goal/task in natural language and have the robot interpret and determine how to carry out this task are huge. This would make robots more accessible generally, since it would reduce the learning curve for operating them. A patient in a hospital would likely find it much easier to simply tell the robot 'I need water', rather than having to find the 'water' option on some sort of touchscreen interface. It also would support situations where the instructions being given are too complex for this kind of interface--one example of this is human-robot teaming, in which groups of humans and robots cooperate to achieve tasks in situations such as disaster scenarios\cite{Kruijff-Korbayova:2015aa}. Communication in natural language can facilitate organisation where all of the humans are working with all of the robots, rather than an individual human having control over one or many robots. Situations like this require complex understanding of natural language instructions. \\
For a household agent, it would be great to be able to interact with a robot in a similar way to how one would with another person. 
\begin{verbatim}
Human: Go get my computer from the office. 
Robot navigates to the office, finds the computer, brings it back. 
Human: Were my keys in the office?
Robot: I don't know, I'll go check. 
Robot navigates to the office, finds the keys. Returns.
Robot: Yes, they are on the bookshelf. 
Human: Thanks. 
\end{verbatim}
For this project, I'll be considering a simpler dialogue stub: the human asks a question, and the robot answers. % SD 2021-05-01 14:20:52 +0200: This is a more compressed version of the same dialogue. In our scenario the robot goes to the desired object but it does not retrieve it. Object retrieval by a robot would be a different task. Replace "dialogue stub" with "conversation". 
\begin{verbatim}
Q: Where is the computer?
A: In the office. 
\end{verbatim}
% \\

This halftime report, as an in-progress version of the thesis report, gives an outline of methods and results so far, as well as planned methods for in-progress and planned tasks. 
